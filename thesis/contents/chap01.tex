% !TeX root = ../bjfuthesis-main.tex

\chapter{数据采集与预处理}

% ============================================================
% 本章核心内容:农村电压数据的采集、特征定义、预处理流程
% 对应代码模块:data_provider/, dataset/RuralVoltage/
% ============================================================

\section{农村电网数据采集体系}

% TODO: 描述农村电网数据采集的整体架构
% - 智能电表、配电自动化终端、气象站等数据源
% - 数据采集频率、通信方式

\subsection{数据采集设备与通信架构}

% 待补充具体内容

\subsection{数据采集频率与存储策略}

% 待补充具体内容


\section{电压数据特征定义}

% 对应代码:dataset/RuralVoltage/generate_sample_data.py 中的17维特征

\subsection{三相电压电流特征}

% 特征:Va, Vb, Vc (三相电压), Ia, Ib, Ic (三相电流)
% 取值范围:电压200-240V,电流10-20A

\subsection{功率指标特征}

% 特征:P (有功功率), Q (无功功率), S (视在功率), PF (功率因数)

\subsection{电能质量特征}

% 特征:THD_Va, THD_Vb, THD_Vc (谐波失真率), V_unbalance, I_unbalance (不平衡因子)
% 标准参考:GB/T 12325-2008

\subsection{运行工况特征}

% 特征:Freq (频率 50Hz), 时间戳等


\section{数据预处理方法}

% 对应代码:data_provider/data_loader.py 中的 StandardScaler

\subsection{缺失值与异常值处理}

% TODO: 描述数据清洗策略

\subsection{数据标准化}

% StandardScaler 标准化方法
% z-score 标准化公式

\subsection{滑动窗口采样}

% 对应代码:seq_len 参数,滑动窗口生成时序样本
% 窗口长度选择依据


\section{异常类型定义与标注}

% 对应代码:generate_sample_data.py 中的5种异常类型

\subsection{电压异常分类}

% 5种异常类型:
% 1. Undervoltage (欠压)
% 2. Overvoltage (过压)
% 3. Voltage_Sag (电压骤降)
% 4. Harmonic (谐波畸变)
% 5. Unbalance (三相不平衡)

\subsection{异常注入方法}

% 如何在正常数据中注入异常
% anomaly_ratio 参数

\subsection{标签格式与评估数据集划分}

% train.csv, test.csv, test_label.csv 的组织方式


\section{本章小结}

% 总结数据采集与预处理的关键环节
