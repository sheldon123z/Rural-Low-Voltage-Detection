% !TeX root = ../bjfuthesis-main.tex

\chapter{实验设计与结果分析}

% ============================================================
% 本章核心内容:实验环境、数据集、模型对比、消融实验
% 对应代码模块:run.py, scripts/, 实验结果
% ============================================================

\section{实验环境配置}

\subsection{硬件环境}

% GPU型号、内存、CPU等

\subsection{软件环境}

% Python版本、PyTorch版本
% 依赖包:requirements.txt

\subsection{超参数设置}

% 对应代码:run.py 的命令行参数
% 表格:关键超参数及其取值
% seq_len, d_model, e_layers, top_k, batch_size, learning_rate 等


\section{实验数据集}

\subsection{标准异常检测数据集}

% PSM, MSL, SMAP, SMD, SWAT
% 数据集规模、特征维度、异常比例

\subsection{农村电压数据集构建}

% RuralVoltage 数据集
% 17维特征定义
% 训练集/测试集划分
% 5种异常类型分布


\section{评估指标体系}

% 对应代码:utils/voltage_metrics.py

\subsection{点级评估指标}

% Accuracy, Precision, Recall, F1-score
% 公式定义

\subsection{点调整后指标}

% Point Adjustment 策略说明
% 调整后的 Precision, Recall, F1

\subsection{其他指标}

% ROC-AUC, PR-AUC(可选)


\section{对比实验}

\subsection{基线模型选择}

% 15个对比模型
% TimesNet, DLinear, PatchTST, Transformer, iTransformer 等
% 表格:各模型的参数量和计算复杂度

\subsection{标准数据集实验结果}

% 在 PSM 等数据集上的结果
% 表格:各模型的 Accuracy, Precision, Recall, F1

\subsection{农村电压数据集实验结果}

% 在 RuralVoltage 数据集上的结果
% 表格:各模型性能对比


\section{消融实验}

\subsection{VoltageTimesNet vs TimesNet}

% 预设周期策略的效果
% 时域平滑层的效果

\subsection{序列长度影响分析}

% 不同 seq_len 的效果对比
% 图表:seq_len vs F1-score

\subsection{模型深度影响分析}

% 不同 e_layers 的效果对比

\subsection{周期数影响分析}

% 不同 top_k 的效果对比


\section{效率分析}

\subsection{训练时间对比}

% 各模型训练耗时

\subsection{推理速度对比}

% 各模型推理速度(样本/秒)

\subsection{模型参数量对比}

% 轻量化部署可行性分析


\section{案例分析}

\subsection{典型异常检测案例}

% 可视化:正常样本 vs 异常样本的重构误差
% 各类异常的检测效果展示

\subsection{误检与漏检分析}

% 分析模型的局限性


\section{本章小结}

% 总结实验结论和主要发现
