% !TeX root = ../bjfuthesis-main.tex

% 中英文摘要和关键字

\begin{abstract}
随着"乡村振兴"战略的深入推进和农村电气化水平的显著提升,农村电网的稳定运行对于保障农业生产、改善农民生活质量及促进农村经济发展至关重要。然而,由于农村地区固有的供电半径长、负荷分散且波动性大、电网基础设施相对薄弱等特点,"低电压"问题已成为制约农村电网供电质量和可靠性的核心瓶颈之一。

本研究旨在通过引入先进的时序预测技术,构建一个主动、智能的农村电网低电压异常监管平台。研究的核心在于利用深度学习模型,特别是长短期记忆网络(LSTM)及其混合变体,对电网关键节点的电压数据进行高精度预测,并在此基础上建立一套基于预测误差的动态异常检测框架。

为实现此目标,本研究首先系统梳理了农村电网低电压问题的成因及传统治理方法的局限性。在此基础上,重点研究并实现了多种深度学习预测模型,包括标准LSTM模型、CNN-LSTM混合模型以及GRU模型。同时,研究设计了基于预测值与实际值残差分析的动态阈值检测机制,并融合日历特征、气象特征以及分布式光伏出力等信息,以增强预测的准确性。

实证分析表明,CNN-LSTM混合模型在综合性能上表现最优,而GRU模型则在保持较高精度的同时展现出显著的效率优势。最终,本研究提出了一个集数据采集、处理、预测、预警与可视化于一体的农村电网低电压异常监管平台总体架构,为数字乡村和智慧能源建设提供技术支撑。

  \bjfusetup{
    keywords = {农村电网, 低电压, 异常检测, 时序预测, 深度学习},
  }
\end{abstract}

\begin{abstract*}
With the deep advancement of the "Rural Revitalization" strategy and the significant improvement of rural electrification level, the stable operation of rural power grids is crucial for ensuring agricultural production, improving farmers' quality of life, and promoting rural economic development. However, due to the inherent characteristics of rural areas, including long power supply radius, dispersed and fluctuating loads, and relatively weak grid infrastructure, the "low voltage" problem has become one of the core bottlenecks restricting the power supply quality and reliability of rural power grids.

This research aims to construct a proactive and intelligent low voltage anomaly monitoring platform for rural power grids by introducing advanced time series prediction technology. The core of the research lies in utilizing deep learning models, particularly Long Short-Term Memory networks (LSTM) and their hybrid variants, to achieve high-precision prediction of voltage data at key grid nodes, and establishing a dynamic anomaly detection framework based on prediction error analysis.

To achieve this goal, this research first systematically analyzes the causes of low voltage problems in rural power grids and the limitations of traditional governance methods. On this basis, multiple deep learning prediction models are studied and implemented, including standard LSTM models, CNN-LSTM hybrid models, and GRU models. Meanwhile, a dynamic threshold detection mechanism based on residual analysis between predicted and actual values is designed, incorporating calendar features, meteorological features, and distributed photovoltaic output information to enhance prediction accuracy.

Empirical analysis shows that the CNN-LSTM hybrid model performs best in terms of comprehensive performance, while the GRU model demonstrates significant efficiency advantages while maintaining high accuracy. Finally, this research proposes an overall architecture for a rural power grid low voltage anomaly monitoring platform that integrates data collection, processing, prediction, early warning, and visualization, providing technical support for digital village and smart energy construction.

  \bjfusetup{
    keywords* = {Rural power grid, Low voltage, Anomaly detection, Time series prediction, Deep learning},
  }
\end{abstract*}
