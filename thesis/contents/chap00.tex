% !TeX root = ../bjfuthesis-main.tex

\chapter{引言}

\section{研究背景和意义}

\subsection{农村电网低电压问题现状}

电能作为现代社会不可或缺的基础能源,其稳定可靠的供应是支撑国民经济发展和人民生活水平提升的基石。近年来,中国在"乡村振兴"和"新基建"战略的指引下,农村地区的电气化进程取得了长足的进步。各类家用电器的普及、电动汽车下乡、农产品加工业的发展以及"煤改电"等清洁取暖工程的推广,使得农村用电负荷呈现出快速增长和结构多样化的新态势\cite{ndrc2022}。然而,与快速增长的用电需求相比,农村电网(简称"农网")的基础设施建设和运维管理水平相对滞后,导致了一系列电能质量问题,其中\textbf{低电压问题}尤为突出,已成为影响农网供电可靠性和安全性的主要"顽疾"\cite{zhang2021}。

农村电网低电压,通常指用户终端电压持续低于国家规定标准(GB/T 12325-2008《电能质量 供电电压偏差》规定220V单相供电电压偏差为+7\%至-10\%)的现象。该问题在农忙季节、节假日返乡高峰以及夏季、冬季的极端天气条件下表现得尤为严重。其产生的原因是多方面的,主要可以归结为以下几点:

\begin{enumerate}
    \item \textbf{电网结构薄弱}:大部分农村地区地广人稀,配电网络呈现出典型的辐射状结构,供电半径过长。过长的供电线路导致线路阻抗增大,电压和功率损耗随之增加,尤其是在线路末端,电压降问题十分显著\cite{li2020}。
    \item \textbf{负荷特性复杂}:农村负荷具有明显的时空不均衡性。时间上,负荷高峰集中在早晚炊事和夜间时段,呈现"双峰"或"多峰"特性;节假日期间,返乡人员剧增导致瞬时负荷冲击大。空间上,居民点分散,负荷密度低,难以实现有效的负荷调配与平衡。
    \item \textbf{设备配置不足}:部分地区的配电变压器容量不足或布点不合理,无法满足高峰时段的用电需求,导致过载运行,加剧了电压下降。同时,无功补偿装置的缺失或配置不当,使得电网无功潮流过大,进一步恶化了电压质量\cite{chen2019}。
    \item \textbf{新能源渗透影响}:随着分布式光伏等新能源在农村地区的大规模接入,其固有的间歇性和波动性给电网带来了新的挑战。例如,在日间光照充足时,光伏系统的大量出力可能导致局部电网电压抬升,甚至出现过电压现象;而当云层遮挡或日落时,光伏出力骤减,又可能引发电压的剧烈波动或下跌,增加了电压控制的复杂性\cite{wang2022pv}。
\end{enumerate}

低电压问题对农村社会经济和居民生活造成了多重负面影响。对居民而言,电压过低会导致照明设备亮度不足、家用电器(特别是空调、冰箱等电机类负载)启动困难、运行效率降低甚至损坏,严重影响生活质量。对农业生产而言,会影响灌溉设备、农产品加工机械等动力设备的正常运行,耽误农时,造成经济损失。此外,长期低电压运行还会增加线路损耗,降低电网的经济效益,并可能引发保护装置误动,威胁电网的安全稳定运行。

传统的低电压治理措施,如更换大截面导线、新增或更换大容量变压器、安装无功补偿装置等,虽然在一定程度上能够缓解问题,但这些方法多属于被动的"亡羊补牢"式改造,存在投资巨大、施工周期长、规划难度大等问题。更重要的是,它们缺乏对负荷变化的预见性,难以适应当前农村电网动态、多变的新常态。因此,亟需一种更为智能、高效、经济的解决方案,实现对低电压问题的\textbf{主动监测、提前预警和精准治理}。

\subsection{时序预测技术发展}

随着物联网、大数据和人工智能技术的飞速发展,基于数据驱动的时序预测技术为解决上述挑战提供了全新的思路。时序预测旨在通过分析历史数据序列,挖掘其内在的演变规律和模式,从而对未来的数值进行预测。在电力系统中,负荷预测、电价预测和新能源出力预测等任务已广泛应用时序预测技术,并取得了显著成效\cite{hong2016}。

近年来,以\textbf{长短期记忆网络(Long Short-Term Memory, LSTM)}为代表的循环神经网络(Recurrent Neural Network, RNN)在时序数据建模领域取得了突破性进展。相比于传统的统计学模型(如ARIMA)和常规的机器学习模型(如支持向量机),LSTM通过其独特的门控机制(输入门、遗忘门、输出门),能够有效学习和记忆时间序列中的长期依赖关系,解决了传统RNN在处理长序列时容易出现的梯度消失或爆炸问题\cite{hochreiter1997}。这一特性使其在处理具有复杂周期性、趋势性和非线性特征的电力系统数据时表现出卓越的性能。

基于LSTM的成功,一系列更为先进的混合深度学习模型被相继提出,进一步提升了预测的精度和鲁棒性:

\begin{itemize}
    \item \textbf{CNN-LSTM混合模型}:该模型首先利用卷积神经网络(CNN)强大的特征提取能力,自动捕捉输入的多维数据(如电压、电流、气象数据等)之间的局部关联和空间特征,然后将提取的特征序列输入LSTM网络进行时序建模。这种"先空间,后时间"的策略能够更全面地理解数据,尤其适用于处理含噪声或多变量输入的场景\cite{alhussein2020}。
    \item \textbf{融合图神经网络(GNN)的模型}:电力系统本质上是一个复杂的物理网络,其运行状态不仅受自身时间动态的影响,还受到网络拓扑结构的约束。图神经网络(如GCN、GIN)能够直接对图结构数据进行建模,捕捉节点间的相互影响。将GNN与LSTM等时序模型结合,可以构建\textbf{时空联合预测模型},从时间和空间两个维度同时对电网状态进行建模,有望实现对异常事件的更精准预测和根源定位\cite{jobe2024}。
\end{itemize}

这些先进的时序预测技术的发展,使得对电网关键节点的电压进行高精度、多步长的实时预测成为可能。基于准确的电压预测,可以构建一个\textbf{基于预测误差的异常检测框架}。其核心思想是:通过比较未来的电压预测值与届时采集到的实际值,计算两者之间的残差(Residual)。在正常运行状态下,该残差应在一个较小的范围内波动;一旦残差显著超出预设的动态阈值,即可判定发生了异常事件。这种"预测-比对"的间接检测模式,能够将异常检测问题转化为一个监督学习下的预测问题,从而实现对罕见、突发性电压骤降或设备故障的灵敏捕捉。

\subsection{研究目的}

基于上述背景,本研究的核心目的在于,综合运用先进的深度学习时序预测技术和大数据分析方法,研究并构建一个面向农村电网的低电压异常智能监管平台。具体研究目的包括:

\begin{enumerate}
    \item \textbf{构建高精度电压预测模型}:针对农村电网电压数据特性,研究并实现多种深度学习预测模型(包括标准LSTM、CNN-LSTM混合模型及GRU),并融合气象、日历、光伏出力等多源异构数据,实现对关键节点未来电压水平的精确预测。
    \item \textbf{设计动态异常检测框架}:基于电压预测结果,建立一套基于预测误差残差分析的异常检测机制。研究动态阈值的自适应调整策略,以有效区分正常波动与真实异常,提高预警的准确性和及时性。
    \item \textbf{探索模型轻量化与部署}:对比分析LSTM与GRU等模型的性能与效率,评估不同模型在资源受限的边缘计算终端上进行部署的可行性,为平台的实际应用提供技术选型依据。
    \item \textbf{提出一体化平台架构}:设计一个集数据接入、模型训练、实时预测、异常预警和结果可视化于一体的综合性监管平台技术方案,为实现农村电网的智能化、精细化管理提供一套完整的解决方案。
\end{enumerate}

\subsection{研究意义}

本研究具有重要的理论价值和广阔的应用前景,其意义主要体现在以下几个方面:

\begin{itemize}
    \item \textbf{理论意义}:本研究将深度学习时序预测的前沿技术与电力系统异常检测的实际需求相结合,探索了CNN-LSTM、LSTM-GNN等时空混合模型在电网电压预测与异常定位中的应用,丰富了智能电网状态感知的理论体系。同时,对基于预测的动态阈值异常检测方法进行深入研究,为解决小样本、非均衡场景下的异常检测问题提供了新的思路。
    \item \textbf{技术意义}:研究成果将形成一套完整的农村电网低电压预测与异常检测技术方案,包括数据处理、特征工程、模型构建、训练优化及评估的全流程。通过对比不同模型的效能,为特定场景下的模型选择提供科学依据。特别是对轻量化模型(GRU)应用潜力的探讨,有助于推动AI算法在电网边缘侧的落地应用。
    \item \textbf{应用价值}:本研究构建的智能监管平台能够变被动的故障响应为主动的预测性维护,使电网运检部门能够提前获知潜在的低电压风险,及时采取负荷调整、无功补偿投切、通知用户错峰用电等干预措施,从而有效避免或减轻低电压事件的负面影响。这不仅能显著提升农村地区的供电质量和可靠性,改善用户用电体验,还能通过优化电网运行方式降低线路损耗,提高运维效率和经济效益,为"数字乡村"和"智慧能源"建设提供坚实的技术保障。
\end{itemize}

\section{研究内容与方法}

\subsection{研究内容}

为实现上述研究目标,本论文将围绕以下核心内容展开:

\begin{enumerate}
    \item \textbf{农村电网低电压问题与时序预测理论综述}:深入剖析农村电网低电压的形成机理、危害及传统治理方法的优缺点。系统梳理时序预测技术的发展脉络,重点介绍LSTM、GRU、CNN、GNN等深度学习模型的基本原理及其在电力系统中的研究与应用现状。
    \item \textbf{多源数据融合与特征工程}:研究如何有效融合电网运行数据(电压、负荷)、外部环境数据(气象、日历)以及新能源出力数据(光伏)。设计一套全面的特征工程方案,包括周期性特征的编码、外部影响因素的量化以及数据标准化等,为后续模型构建奠定基础。
    \item \textbf{基于深度学习的电压预测模型构建}:
    \begin{itemize}
        \item \textbf{标准LSTM模型}:构建一个基准LSTM模型,用于捕捉电压时间序列的长期依赖关系。
        \item \textbf{CNN-LSTM混合模型}:设计并实现一个CNN-LSTM混合模型,利用CNN提取多维输入的空间特征,再由LSTM进行时序预测。
        \item \textbf{GRU模型}:构建GRU模型,作为LSTM的对比,评估其在预测精度和计算效率上的表现。
    \end{itemize}
    \item \textbf{基于预测误差的动态异常检测框架设计}:
    \begin{itemize}
        \item \textbf{残差计算}:基于预测模型输出的预测值与真实值,计算两者之间的残差序列。
        \item \textbf{动态阈值}:研究基于残差序列统计特性(如均值、标准差)的动态阈值生成算法,使其能够自适应于电网不同的运行状态。
        \item \textbf{异常判定}:当实时残差超过动态阈值时,系统判定为异常并触发预警。
    \end{itemize}
    \item \textbf{实证分析与模型对比}:
    \begin{itemize}
        \item \textbf{数据集构建}:构建一个能够模拟真实农村电网运行特性的合成数据集,包含丰富的正常运行模式和多样的异常事件。
        \item \textbf{对比实验}:在构建的数据集上,对所提出的LSTM、CNN-LSTM、GRU等模型进行全面的训练和测试。
        \item \textbf{性能评估}:从电压预测精度(MAE, RMSE, MAPE)和异常检测性能(准确率, 精确率, 召回率, F1分数)两个维度,对各模型进行定量评估和对比分析。同时,比较不同模型的训练和推理时间,评估其效率。
    \end{itemize}
    \item \textbf{监管平台总体架构设计}:基于研究成果,设计一个功能完备的农村电网低电压异常监管平台。详细阐述平台的功能模块划分、技术架构、数据流程和交互界面设计,为系统的工程化实现提供蓝图。
\end{enumerate}

\subsection{研究思路}

本研究将遵循"理论研究-模型构建-实验验证-平台设计"的技术路线。研究首先从理论层面出发,明确问题的背景、意义和研究现状。随后,进入数据准备阶段,构建用于模型训练和验证的数据集。核心阶段是模型构建与实现,将分别开发多种深度学习预测模型。紧接着,利用训练好的模型进行电压预测,并在此基础上实施异常检测。之后,通过全面的对比实验来验证和评估所提方法的有效性。最后,在总结研究成果的基础上,完成监管平台的架构设计和论文的最终撰写。

\subsection{研究方法}

本研究将综合采用以下几种研究方法:

\begin{itemize}
    \item \textbf{文献研究法}:广泛查阅国内外关于农村电网、低电压治理、时序预测、深度学习和异常检测等领域的学术论文、技术报告和行业标准,为本研究提供理论基础和技术借鉴。
    \item \textbf{建模与仿真法}:利用Python编程语言及其相关的深度学习框架(如PyTorch),构建和实现所述的各类预测与检测模型。通过设计和生成合成数据集,模拟真实的电网运行环境,对模型进行训练和测试。
    \item \textbf{对比实验法}:设计严格的对比实验方案,在相同的实验条件下,对不同模型的性能进行横向比较,客观评估各方法的优劣和适用场景。
    \item \textbf{案例分析法}:在实验结果的分析中,选取典型的正常和异常时段,对模型的预测行为和检测结果进行深入剖析,以直观展示模型的工作机制和有效性。
\end{itemize}

\section{论文结构与创新}

\subsection{论文结构与框架}

本论文共分为六章,具体结构安排如下:

\begin{itemize}
    \item \textbf{第一章:引言}。介绍研究的背景、意义、国内外研究现状,明确研究目的、内容和方法,并概述论文的结构与创新点。
    \item \textbf{第二章:理论基础与文献综述}。详细阐述农村电网运行特点、低电压问题定义,以及时序预测和深度学习相关理论,并对相关领域的研究进展进行综述。
    \item \textbf{第三章:系统设计与关键技术}。重点介绍本研究提出的低电压监管系统总体架构,并详细阐述数据处理、特征工程、预测模型构建、异常检测框架等关键技术的设计原理和实现细节。
    \item \textbf{第四章:平台开发与实现}。本章将结合第三章的设计,展示平台的具体开发过程,包括前端界面、后端功能以及算法集成,并进行系统测试与验证。
    \item \textbf{第五章:实证分析与案例研究}。介绍实验所用数据集的构建过程,展示各模型的训练与评估结果,通过图表进行详细的性能对比,并结合具体案例分析平台的运行效果。
    \item \textbf{第六章:结论与展望}。总结全文的研究工作和主要贡献,指出研究存在的局限性,并对未来的研究方向进行展望。
\end{itemize}

\subsection{论文创新点}

本研究的主要创新点体现在以下几个方面:

\begin{enumerate}
    \item \textbf{融合多维特征的时空联合预测视角}:与多数仅关注电压单一时间序列的预测方法不同,本研究不仅融合了气象、日历等多维外部特征,还前瞻性地探讨了融合电网拓扑结构的GNN模型,为实现从"单点预测"到"全网态势感知"的跨越提供了理论和技术路径。
    \item \textbf{面向农村电网特性的异常检测框架}:针对农村电网负荷波动大、异常模式多样的特点,设计了基于预测误差的动态阈值异常检测方法。该方法能够自适应地调整灵敏度,并着重考虑了如何区分光伏出力波动等"正常"扰动与设备故障等"真实"异常,提高了检测的精准度和实用性。
    \item \textbf{模型实用性与效率的综合考量}:本研究不仅追求预测和检测的最高精度,还通过引入GRU模型与LSTM进行对比,系统地评估了不同模型在计算效率和资源消耗上的差异,为模型在边缘计算设备上的轻量化部署提供了实证依据,增强了研究成果的工程应用潜力。
    \item \textbf{系统化的平台设计与验证}:本研究不局限于算法层面的探讨,而是从系统工程的角度出发,提出了一个从数据到应用的全链路解决方案,并构建了高度仿真的数据集对整个方案进行了端到端的验证,确保了研究的完整性和可行性。
\end{enumerate}
