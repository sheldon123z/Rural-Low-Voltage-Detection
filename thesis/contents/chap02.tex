% !TeX root = ../bjfuthesis-main.tex

\chapter{基于TimesNet的电压异常检测算法}

% ============================================================
% 本章核心内容:TimesNet模型原理、改进方法、异常检测框架
% 对应代码模块:models/TimesNet.py, models/VoltageTimesNet.py
% ============================================================

\section{时序异常检测方法概述}

\subsection{基于重构的异常检测原理}

% 核心思想:模型学习正常模式,通过重构误差检测异常
% 对应代码:exp/exp_anomaly_detection.py 中的 MSELoss

\subsection{深度学习时序模型发展}

% RNN → LSTM → Transformer → TimesNet 的演进
% 各类模型的优缺点对比


\section{TimesNet模型原理}

% 对应代码:models/TimesNet.py

\subsection{模型整体架构}

% 输入嵌入 → TimesBlock堆叠 → LayerNorm → 投影层
% 图示:模型架构图

\subsection{FFT周期发现机制}

% 对应代码:FFT_for_Period 函数
% 核心创新:将1D时序转换为2D张量处理
% 公式:FFT变换、频率提取、周期计算

\subsection{2D卷积特征提取}

% 对应代码:Inception_Block_V1
% 1D → 2D reshape → 2D Conv → 1D reshape
% Inception多尺度卷积核设计

\subsection{自适应周期聚合}

% 多个周期结果的加权融合
% Softmax权重计算


\section{VoltageTimesNet改进设计}

% 对应代码:models/VoltageTimesNet.py
% 针对农村电压数据的定制化改进

\subsection{预设周期与FFT混合策略}

% 电网固有周期:1min, 5min, 15min, 1h
% 70% FFT动态发现 + 30% 预设周期

\subsection{时域平滑层设计}

% 深度可分离1D卷积
% 抑制高频噪声


\section{基于重构误差的异常检测框架}

% 对应代码:exp/exp_anomaly_detection.py 的 test() 方法

\subsection{重构误差计算}

% MSE Loss 计算公式
% 逐时间步的误差序列

\subsection{动态阈值设定}

% percentile 方法
% anomaly_ratio 参数的作用

\subsection{点调整评估策略}

% Point Adjustment 算法
% 对应代码:adjustment() 函数
% 适用于时序异常检测的评估方式


\section{模型训练与优化}

\subsection{损失函数设计}

% MSELoss 重构损失

\subsection{优化器与学习率策略}

% Adam 优化器
% 学习率衰减策略

\subsection{早停机制}

% 对应代码:utils/tools.py 的 EarlyStopping
% patience 参数


\section{本章小结}

% 总结TimesNet算法原理和改进要点
